%%%%%%%%%%%%%%%%%%%%%%%%%%%%%%%%%%%%%%%
\thesisspacing % CHAPTER
% COPY THEM IN ANY NEW CHAPTER
%%%%%%%%%%%%%%%%%%%%%%%%%%%%%%%%%%%%%%%

\subsection{Model Evaluation and Backtesting}

The evaluation of the Convolutional Neural Network (CNN) model was conducted through a two-step process that involved both classification performance assessment and a comprehensive backtesting strategy. The aim was to assess the model's effectiveness in predicting market movements and to evaluate its practical applicability within a real-world trading environment.

\subsubsection{Model Evaluation}

The CNN model, designed as a classifier, was initially evaluated using standard classification metrics: accuracy, precision, recall, and F1 score. These metrics provided a comprehensive overview of the model's performance in distinguishing between various market conditions, such as bullish, bearish, or neutral trends.

\begin{itemize}
    \item \textbf{Accuracy} was utilized to measure the proportion of correct predictions out of the total number of predictions, offering a broad sense of the model's overall performance.
    \item \textbf{Precision} was calculated to determine the proportion of true positive predictions out of all positive predictions made by the model, indicating how effectively the model identifies market conditions that it forecasts to occur.
    \item \textbf{Recall} was assessed to understand the model's sensitivity in correctly identifying all actual instances of a specific market condition.
    \item \textbf{F1 score}, the harmonic mean of precision and recall, was used to balance the trade-off between these two metrics, particularly in financial market prediction contexts where false positives and false negatives have different implications on trading strategies.
\end{itemize}

These evaluation metrics provided a detailed assessment of the CNN models likely to perform well in practical trading scenarios. Based on these evaluations, the models that demonstrated the best balance between accuracy, precision, recall, and F1 score were selected for the subsequent phase: backtesting.

\subsubsection{Backtesting Strategy}

Following the initial model evaluation, the selected CNN models were subjected to rigorous backtesting to assess their performance within a simulated trading environment. The backtesting covered a period from June 2019 to June 2024, using historical market data to evaluate the models' predictions and trading decisions.

The backtesting strategy employed a weekly rebalancing approach, consistent with the lookback periods used during the model training phase. This strategy, inspired by the methodology detailed in Kelly's and Jiangs's paper \cite{jiang2023reimagining}, ensures coherence between the training and testing phases, thereby providing a robust framework for evaluating the model's predictive capabilities over time. The weekly rebalancing strategy allows the model to adjust its positions based on updated signals, reflecting a dynamic trading approach that is responsive to evolving market conditions.

The CNN model generated signals indicating whether to go long, short, or hold a position based on its predictions. A "long" signal indicated a positive market outlook, prompting the strategy to purchase and hold the asset, while a "short" signal suggested a negative outlook, prompting the strategy to sell or short the asset. A "hold" signal indicated a neutral outlook, suggesting no changes to the current position. This tripartite strategy is designed to capture market opportunities while managing downside risk, leveraging the CNN model's predictive power to guide trading decisions.

The backtesting was conducted using QSTrader, an open-source framework for implementing systematic trading strategies. QSTrader enabled a comprehensive analysis of the strategy's performance relative to a buy-and-hold strategy of the S\&P 500 index over the same period. The buy-and-hold strategy served as a benchmark, representing a passive investment approach commonly employed in the market.

To evaluate the effectiveness of the CNN-driven trading strategy, several key performance metrics were analyzed:

\begin{itemize}
    \item \textbf{Cumulative Return}: The total return of the portfolio over the backtesting period, providing a direct comparison between the growth of the portfolio under the CNN strategy and the buy-and-hold strategy.
    \item \textbf{Sharpe Ratio}: A measure of risk-adjusted return, calculated as the ratio of the portfolio's excess return over the risk-free rate to its standard deviation. This metric was used to assess how well the CNN-driven strategy compensated for risk relative to the benchmark.
    \item \textbf{Maximum Drawdown}: The maximum observed loss from a peak to a trough of the portfolio, before a new peak is attained. This metric was crucial for evaluating the risk exposure of the CNN strategy compared to the buy-and-hold benchmark.
    \item \textbf{Volatility}: The standard deviation of the portfolio's returns, indicating the level of risk or uncertainty associated with the strategy's performance.
\end{itemize}

The backtesting results demonstrated that the CNN-driven strategy, with its dynamic weekly rebalancing and responsiveness to market signals, outperformed the buy-and-hold strategy of the S\&P 500 in terms of cumulative return and risk-adjusted performance. The strategy successfully captured significant market trends, both upward and downward, by dynamically adjusting its positions based on the CNN model's signals. The CNN strategy exhibited a higher Sharpe ratio, indicating superior risk-adjusted returns compared to the benchmark. Additionally, the maximum drawdown of the CNN strategy was lower than that of the buy-and-hold approach, suggesting that the model was effective in mitigating downside risk during market downturns.

By benchmarking the CNN-driven strategy against a buy-and-hold approach, the study effectively highlighted the potential advantages of utilizing advanced machine learning techniques for financial market prediction and portfolio management. The results underscore the practical utility of CNN models in developing systematic trading strategies that are not only predictive but also adaptable to varying market conditions. This research contributes to the expanding body of literature on the application of deep learning in financial markets, offering insights into how such models can be leveraged to enhance portfolio performance and manage investment risk effectively.
