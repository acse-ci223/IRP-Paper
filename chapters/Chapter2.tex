\doublespacing % Do not change - required

\chapter{Methodology}
\label{ch2}

%%%%%%%%%%%%%%%%%%%%%%%%%%%%%%%%%%%%%%%
% IMPORTANT
\begin{spacing}{1} %THESE FOUR
\minitoc % LINES MUST APPEAR IN
\end{spacing} % EVERY
\thesisspacing % CHAPTER
% COPY THEM IN ANY NEW CHAPTER
%%%%%%%%%%%%%%%%%%%%%%%%%%%%%%%%%%%%%%%

% \section{Training}
% %%%%%%%%%%%%%%%%%%%%%%%%%%%%%%%%%%%%%%%
\thesisspacing % CHAPTER
% COPY THEM IN ANY NEW CHAPTER
%%%%%%%%%%%%%%%%%%%%%%%%%%%%%%%%%%%%%%%

\subsection{Data Preprocessing and Integration}

The preprocessing phase was a critical step in preparing the data for effective training of the Convolutional Neural Network (CNN) model. Given the nature of financial time-series data and the need to convert it into formats suitable for deep learning, a rigorous and systematic approach was adopted.

Initially, the data underwent a comprehensive cleaning process to address inconsistencies, missing values, and outliers. As the data was sourced from multiple repositories—namely CRSP, Kaggle, and Yahoo! Finance—standardization across datasets was paramount. Missing values were handled using techniques such as forward and backward filling to maintain the temporal continuity of the data, while outliers were identified and removed using statistical methods like Z-score analysis and interquartile range (IQR) filtering. This step was essential to ensure that the data fed into the CNN models was both reliable and representative of typical market behaviors.

Following the cleaning process, normalization was applied to standardize the range of the OHLC (Open, High, Low, Close) data. Normalization is crucial in neural network models to ensure that all input features contribute equally to the learning process. This typically involved scaling the OHLC data to a range between 0 and 1, which helps in achieving faster convergence during the training phase and avoids biases that could arise from differing magnitudes in raw data values.

A significant aspect of the preprocessing phase was the transformation of the normalized OHLC data into 64x64 pixel images, specifically candlestick charts, which serve as the primary input format for the CNN models. The transformation process involved converting sequential OHLC data points into a series of candlestick charts, capturing the temporal dynamics and price movements over fixed intervals. These images were then stored in .npy files, an efficient format for handling large-scale image datasets within NumPy arrays, enabling streamlined data loading and manipulation during the training phase.

This approach of converting OHLC time-series data into visual representations leverages the ability of CNNs to recognize complex spatial patterns, which are indicative of underlying market trends and behaviors. By using a visual representation, the CNN model is better positioned to learn from patterns in the data that are not easily captured through traditional numerical methods, thus enhancing the robustness and accuracy of the predictive model.

\subsection{Model Development and Training}

The development and training of the CNN model were meticulously structured to explore various architectural configurations aimed at maximizing predictive accuracy while minimizing overfitting. The chosen architecture treated the model as a classifier, designed to predict market direction based on the input images derived from OHLC data.

The initial phase of model development involved experimentation with multiple CNN architectures to identify the most effective structure for financial market prediction. This experimentation included testing various combinations of convolutional layers, pooling layers, and activation functions. Key architectural features such as dropout layers were integrated to reduce overfitting by randomly deactivating a subset of neurons during each training iteration. This technique helps to generalize the model, ensuring it does not overly fit the training data at the expense of performance on unseen data.

Residual blocks were also employed to enhance the model's depth and ability to capture complex features. Residual connections facilitate the training of deep networks by allowing gradients to flow more effectively through multiple layers, preventing the vanishing gradient problem commonly encountered in deep learning. This capability is particularly valuable in financial modeling, where deep architectures can uncover complex, non-linear relationships inherent in market data.

In addition to standard CNN layers, the model architecture incorporated Long Short-Term Memory (LSTM) layers to capture sequential dependencies in the time-series data. LSTMs, a type of recurrent neural network (RNN), are well-suited for handling sequences and can retain information across time steps, making them ideal for capturing the temporal dependencies crucial in financial forecasting.

The training dataset, encompassing randomly selected stocks from the U.S. market from the 1990s until 2017, was deliberately chosen to ensure that the model trained on data entirely unrelated to the backtesting period (June 2019 to June 2024). This strategic choice was aimed at preventing any data leakage between the training and testing phases, thereby enhancing the model's generalizability and ensuring that performance metrics reflect true predictive power rather than overfitting to historical data.

The training process involved optimizing the model's parameters through iterative tuning of hyperparameters such as learning rates, batch sizes, and epochs. Optimization algorithms, including Adam and RMSprop, were explored to identify the most effective approach for minimizing the loss function and achieving stable convergence. The model's performance was continuously evaluated using classification metrics such as accuracy, precision, recall, and F1 score, which provided an outline of which configurations were likely to perform best during the subsequent backtesting phase.

By rigorously testing and refining these CNN architectures, the study aimed to identify the model configuration that most effectively captures the complex patterns in financial data, providing a robust tool for predicting market movements and contributing valuable insights to the field of financial analytics.

% \section{Evaluating}
% %%%%%%%%%%%%%%%%%%%%%%%%%%%%%%%%%%%%%%%
\thesisspacing % CHAPTER
% COPY THEM IN ANY NEW CHAPTER
%%%%%%%%%%%%%%%%%%%%%%%%%%%%%%%%%%%%%%%

\subsection{Model Evaluation and Backtesting}

The evaluation of the Convolutional Neural Network (CNN) model was conducted through a two-step process that involved both classification performance assessment and a comprehensive backtesting strategy. The aim was to assess the model's effectiveness in predicting market movements and to evaluate its practical applicability within a real-world trading environment.

\subsubsection{Model Evaluation}

The CNN model, designed as a classifier, was initially evaluated using standard classification metrics: accuracy, precision, recall, and F1 score. These metrics provided a comprehensive overview of the model's performance in distinguishing between various market conditions, such as bullish, bearish, or neutral trends.

\begin{itemize}
    \item \textbf{Accuracy} was utilized to measure the proportion of correct predictions out of the total number of predictions, offering a broad sense of the model's overall performance.
    \item \textbf{Precision} was calculated to determine the proportion of true positive predictions out of all positive predictions made by the model, indicating how effectively the model identifies market conditions that it forecasts to occur.
    \item \textbf{Recall} was assessed to understand the model's sensitivity in correctly identifying all actual instances of a specific market condition.
    \item \textbf{F1 score}, the harmonic mean of precision and recall, was used to balance the trade-off between these two metrics, particularly in financial market prediction contexts where false positives and false negatives have different implications on trading strategies.
\end{itemize}

These evaluation metrics provided a detailed assessment of the CNN models likely to perform well in practical trading scenarios. Based on these evaluations, the models that demonstrated the best balance between accuracy, precision, recall, and F1 score were selected for the subsequent phase: backtesting.

\subsubsection{Backtesting Strategy}

Following the initial model evaluation, the selected CNN models were subjected to rigorous backtesting to assess their performance within a simulated trading environment. The backtesting covered a period from June 2019 to June 2024, using historical market data to evaluate the models' predictions and trading decisions.

The backtesting strategy employed a weekly rebalancing approach, consistent with the lookback periods used during the model training phase. This strategy, inspired by the methodology detailed in Kelly's and Jiangs's paper \cite{jiang2023reimagining}, ensures coherence between the training and testing phases, thereby providing a robust framework for evaluating the model's predictive capabilities over time. The weekly rebalancing strategy allows the model to adjust its positions based on updated signals, reflecting a dynamic trading approach that is responsive to evolving market conditions.

The CNN model generated signals indicating whether to go long, short, or hold a position based on its predictions. A "long" signal indicated a positive market outlook, prompting the strategy to purchase and hold the asset, while a "short" signal suggested a negative outlook, prompting the strategy to sell or short the asset. A "hold" signal indicated a neutral outlook, suggesting no changes to the current position. This tripartite strategy is designed to capture market opportunities while managing downside risk, leveraging the CNN model's predictive power to guide trading decisions.

The backtesting was conducted using QSTrader, an open-source framework for implementing systematic trading strategies. QSTrader enabled a comprehensive analysis of the strategy's performance relative to a buy-and-hold strategy of the S\&P 500 index over the same period. The buy-and-hold strategy served as a benchmark, representing a passive investment approach commonly employed in the market.

To evaluate the effectiveness of the CNN-driven trading strategy, several key performance metrics were analyzed:

\begin{itemize}
    \item \textbf{Cumulative Return}: The total return of the portfolio over the backtesting period, providing a direct comparison between the growth of the portfolio under the CNN strategy and the buy-and-hold strategy.
    \item \textbf{Sharpe Ratio}: A measure of risk-adjusted return, calculated as the ratio of the portfolio's excess return over the risk-free rate to its standard deviation. This metric was used to assess how well the CNN-driven strategy compensated for risk relative to the benchmark.
    \item \textbf{Maximum Drawdown}: The maximum observed loss from a peak to a trough of the portfolio, before a new peak is attained. This metric was crucial for evaluating the risk exposure of the CNN strategy compared to the buy-and-hold benchmark.
    \item \textbf{Volatility}: The standard deviation of the portfolio's returns, indicating the level of risk or uncertainty associated with the strategy's performance.
\end{itemize}

The backtesting results demonstrated that the CNN-driven strategy, with its dynamic weekly rebalancing and responsiveness to market signals, outperformed the buy-and-hold strategy of the S\&P 500 in terms of cumulative return and risk-adjusted performance. The strategy successfully captured significant market trends, both upward and downward, by dynamically adjusting its positions based on the CNN model's signals. The CNN strategy exhibited a higher Sharpe ratio, indicating superior risk-adjusted returns compared to the benchmark. Additionally, the maximum drawdown of the CNN strategy was lower than that of the buy-and-hold approach, suggesting that the model was effective in mitigating downside risk during market downturns.

By benchmarking the CNN-driven strategy against a buy-and-hold approach, the study effectively highlighted the potential advantages of utilizing advanced machine learning techniques for financial market prediction and portfolio management. The results underscore the practical utility of CNN models in developing systematic trading strategies that are not only predictive but also adaptable to varying market conditions. This research contributes to the expanding body of literature on the application of deep learning in financial markets, offering insights into how such models can be leveraged to enhance portfolio performance and manage investment risk effectively.



%%%%%%%%%%%%%%%%%%%%%%%%%%%%%%%%%%%%%%%
\thesisspacing % CHAPTER
% COPY THEM IN ANY NEW CHAPTER
%%%%%%%%%%%%%%%%%%%%%%%%%%%%%%%%%%%%%%%

% \section{Methodology}

This section outlines the technical approach used in developing and testing the Convolutional Neural Network (CNN) model for financial market prediction. The methodology is divided into several components: data acquisition and preprocessing, model development, evaluation and backtesting, and visualization and reporting. Each component is described in detail, highlighting the tools, libraries, and techniques used throughout the development process.

\section{Data Acquisition and Preprocessing}

\subsection{Data Acquisition and Scope}

The initial phase of the project involved the acquisition of high-quality financial time-series data from multiple sources, including the Center for Research in Security Prices (CRSP), Kaggle, and Yahoo! Finance. These datasets provided comprehensive OHLC (Open, High, Low, Close) data across a wide range of securities listed on major U.S. stock exchanges, covering a period from the 1990s to 2017. The choice of this timeframe was deliberate to ensure that the training data was unrelated to the backtesting period (June 2019 to June 2024), thereby minimizing the risk of overfitting.

\subsection{Data Preprocessing and Integration}

Once acquired, the data underwent a rigorous preprocessing phase to ensure quality and suitability for deep learning applications. This phase involved several steps:

\begin{itemize}
    \item \textbf{Data Cleaning}: Handling missing values using forward and backward filling techniques, and removing outliers using statistical methods such as Z-score analysis and interquartile range (IQR) filtering.
    \item \textbf{Normalization}: Standardizing the scale of OHLC data to ensure consistency across the dataset, typically scaling values between 0 and 1.
    \item \textbf{Transformation to Image Format}: The normalized OHLC data was converted into 64x64 pixel candlestick chart images using libraries such as Pandas, PIL, and Plotly. These images were then stored as .npy files for efficient loading and processing during the model training phase.
\end{itemize}

\textbf{Simplified Pseudocode for Data Preparation}:
\begin{verbatim}
Read CSV data files
For each data point:
    Clean and normalize data
    Convert OHLC data to candlestick chart image
    Save image as .npy file
\end{verbatim}

\section{Model Development}

\subsection{CNN Model Development}

The core of the methodology focused on developing a robust CNN model using PyTorch, designed to predict financial market conditions based on visual representations of OHLC data.

\begin{itemize}
    \item \textbf{Model Architecture}: The CNN architecture was constructed with multiple layers, including convolutional layers, pooling layers, dropout layers, and fully connected layers. The architecture was optimized to capture spatial patterns in the candlestick chart images.
    \item \textbf{Training Process}: The model was trained using GPU acceleration (CUDA) to handle large datasets efficiently. Hyperparameters such as learning rates, batch sizes, and epochs were carefully tuned to optimize model performance.
\end{itemize}

\textbf{Simplified Pseudocode for CNN Model Development}:
\begin{verbatim}
Define CNN architecture with multiple layers
Initialize model parameters
For each epoch:
    Load image data
    Forward pass through the network
    Compute loss and gradients
    Update model parameters
Save trained model
\end{verbatim}

\section{Evaluation and Backtesting}

\subsection{Model Evaluation}

The CNN model, developed as a classifier, was evaluated using standard classification metrics to determine its performance in distinguishing between various market conditions:

\begin{itemize}
    \item \textbf{Accuracy}: Proportion of correct predictions out of the total number of predictions.
    \item \textbf{Precision}: Proportion of true positive predictions out of all positive predictions made by the model.
    \item \textbf{Recall}: Model's sensitivity to correctly identifying all actual instances of a specific market condition.
    \item \textbf{F1 Score}: Harmonic mean of precision and recall, balancing the trade-off between these metrics.
\end{itemize}

\subsection{Backtesting Strategy}

Following model evaluation, a comprehensive backtesting strategy was employed to assess the CNN model's performance in a simulated trading environment. The strategy was implemented over a period from June 2019 to June 2024, aligning with the model's lookback periods and following a weekly rebalancing approach.

\begin{itemize}
    \item \textbf{Trading Signals}: The CNN model generated signals to go long, short, or hold based on its predictions.
    \item \textbf{Backtesting Framework}: QSTrader was used to simulate trades and evaluate the strategy's performance against a benchmark buy-and-hold strategy of the S\&P 500 index.
    \item \textbf{Performance Metrics}: Key metrics such as cumulative return, Sharpe ratio, maximum drawdown, and volatility were calculated to assess the effectiveness of the CNN-driven strategy.
\end{itemize}

\textbf{Simplified Pseudocode for Backtesting Strategy}:
\begin{verbatim}
Load trained model
For each backtesting period:
    Generate trading signals (long, short, hold)
    Execute trades in simulated environment
    Calculate performance metrics (return, Sharpe ratio, drawdown)
Compare performance to buy-and-hold benchmark
\end{verbatim}

\section{Visualization and Reporting}

\begin{itemize}
    \item \textbf{Rationale}: To provide a clear and comprehensive visualization of the model's performance and the results of the backtesting strategy.
    \item \textbf{Implementation}: Utilized Matplotlib to generate plots and charts illustrating key performance metrics and outcomes.
\end{itemize}

\textbf{Simplified Pseudocode for Visualization}:
\begin{verbatim}
Collect performance metrics
Generate visual plots for cumulative return, Sharpe ratio, drawdown
Display comparison graphs
Export visual reports
\end{verbatim}

\section{Technical Platform and Implementation Details}

The implementation of the solution was conducted entirely using Python, leveraging a range of open-source libraries and frameworks to facilitate various aspects of data processing, model development, and evaluation. The development environment primarily utilized Visual Studio Code (VSCode), which provided a robust platform for writing, testing, and organizing the Python scripts into modular files. This modular approach allowed for the separation of different functional components, such as data preparation, model training, and backtesting, thereby enhancing maintainability and facilitating iterative development.

For data processing and visualization, libraries such as Pandas, PIL, Plotly, and Matplotlib were employed. Pandas was utilized for data manipulation tasks, including reading CSV files and handling missing values, while PIL and Plotly were used to generate candlestick chart images from the processed OHLC data. Matplotlib was also employed to create additional visualizations, both during the exploratory data analysis phase and for presenting the final results.

The core of the model development was implemented using PyTorch, a popular deep learning framework known for its flexibility and support for dynamic computation graphs. PyTorch, in combination with CUDA, enabled efficient training of the CNN model on GPUs, significantly accelerating the computation and allowing for more extensive hyperparameter tuning. NumPy was also utilized for numerical computations, providing efficient array operations and integration with other libraries.

For the evaluation and backtesting of the CNN model, QSTrader, an open-source framework for systematic trading strategies, was used. QSTrader facilitated the simulation of trading strategies based on the model's outputs and enabled a comprehensive analysis of the model's performance against a benchmark buy-and-hold strategy of the S\&P 500 index. This integration allowed for a rigorous assessment of the model's predictive capabilities in a controlled environment, replicating real-world trading scenarios.

The development process was initially carried out within Jupyter Notebooks to allow for interactive experimentation and visualization of results. Upon achieving satisfactory performance, the code was then refactored into standalone Python scripts for a more production-ready deployment. This transition ensured that the final implementation was optimized for operational use while retaining the flexibility and modularity required for further enhancements and testing.

Overall, the choice of tools and frameworks was guided by their suitability for handling large-scale financial data and their ability to support the iterative development of deep learning models. The combination of Python's ecosystem of libraries with a modular development strategy provided a robust and scalable solution, capable of addressing the complex challenges associated with financial market prediction using CNN architectures.