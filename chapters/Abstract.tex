% Write your abstract here

This project addresses the increasing complexity and volatility in financial markets through the development of advanced analytical tools. Leveraging the theoretical foundations established by Bryan Kelly and Kusuma, we propose refining Convolutional Neural Networks (CNNs) to enhance the prediction of financial asset behaviors. Traditional methods like Auto Regressive Integrated Moving Average (ARIMA) often fail to capture the nonlinear dynamics and intricate patterns present in financial data. By contrast, CNNs are capable of exploiting image representations of time-series data to learn spatial and temporal features, potentially offering superior predictive accuracy. 

In this study, we develop a robust CNN model trained on candlestick chart images generated from a dataset of randomly selected U.S. stocks spanning from the 1990s to 2017. The model's performance was rigorously evaluated through backtesting on the S\&P 500 index over a period from June 2019 to June 2024. The results indicate that the well-trained model achieved a Sharpe ratio of 1.20 and maintained a drawdown of 24.46\%, significantly outperforming a poorly trained model and traditional benchmarks. The model also demonstrated robustness in identifying market anomalies, such as the COVID-19 pandemic and subsequent volatility in 2022, underscoring its adaptability to diverse market conditions. While these findings highlight the potential of CNNs in financial forecasting, challenges such as overfitting and data balancing remain. Future work will explore the integration of more advanced architectures, such as Transformers, and the inclusion of additional data features to further enhance model performance and generalizability across different asset classes.

