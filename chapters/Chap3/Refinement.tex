%%%%%%%%%%%%%%%%%%%%%%%%%%%%%%%%%%%%%%%
\thesisspacing % CHAPTER
% COPY THEM IN ANY NEW CHAPTER
%%%%%%%%%%%%%%%%%%%%%%%%%%%%%%%%%%%%%%%

% Write your text here
\subsection{Initial Configuration and Parameter Tuning}
The initial stage of the project involves setting up the fundamental parameters of our Convolutional Neural Network (CNN) architectures, including layer structures, activation functions, and learning rates, based on established best practices and insights from prior research. This setup retains flexibility for modifications in response to preliminary testing outcomes. Concurrently, parameter tuning is a critical phase where the model's hyperparameters, such as the number of convolutional layers, filter sizes, dropout rates, and learning rates, are meticulously adjusted to enhance the model’s accuracy and processing efficiency.

\subsection{Iterative Testing and Validation}
The robustness and reliability of the model are confirmed through iterative testing and validation. This process involves assessing the model across various scenarios, comparing its predictive outputs against actual market behaviors. Feedback from these evaluations is used to continuously refine the model, making necessary adjustments to improve predictive accuracy and address potential issues like overfitting or underfitting, thereby enhancing the model's overall stability and reliability.
