%%%%%%%%%%%%%%%%%%%%%%%%%%%%%%%%%%%%%%%
\thesisspacing % CHAPTER
% COPY THEM IN ANY NEW CHAPTER
%%%%%%%%%%%%%%%%%%%%%%%%%%%%%%%%%%%%%%%

% Write your text here
\subsection{Project Overview, Strategy, and Data Handling}
This research initiative aims to develop a deep learning model using Convolutional Neural Networks (CNNs) to analyze asset class movements through stacked image inputs. Integrating multiple asset classes within a single image allows the model to discern intricate co-movements among asset classes like stocks, bonds, commodities, and currencies, refining trading decisions. Data for this model will be processed to capture key financial metrics from various asset classes and prepared in image form to highlight significant market trends, with normalization and scaling ensuring consistency across asset types.

\subsection{Analytical Methods and Predictive Decision-Making}
The core of the model's predictive capability lies in its ability to interpret visual patterns from complex image data, leveraging CNN’s robust feature extraction capabilities to decode collective asset responses to market dynamics. Upon development, the model will categorize assets into 'buy', 'hold', or 'sell', based on predicted movements, informing dynamic portfolio adjustments to optimize financial returns. This framework will evolve as the model matures and additional insights are integrated.

\subsection{Model Evaluation, Optimization, and Future Implementation}
The model's efficacy will be assessed through backtesting with historical image data, focusing on precision, recall, and F1-score to evaluate the accuracy of trading decisions. Continuous refinement will include retraining with updated data and parameter adjustments. Subject to successful validation, the model is slated for real-time trading deployment, with future enhancements planned to integrate real-time data feeds and expand the range of asset classes and markets, enhancing the model’s applicability and strategic utility in global financial markets.
