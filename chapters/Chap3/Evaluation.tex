%%%%%%%%%%%%%%%%%%%%%%%%%%%%%%%%%%%%%%%
\thesisspacing % CHAPTER
% COPY THEM IN ANY NEW CHAPTER
%%%%%%%%%%%%%%%%%%%%%%%%%%%%%%%%%%%%%%%

% Write your text here
\subsection{Evaluation Strategy and Benchmarking}
To rigorously assess the effectiveness of our deep learning model in forecasting asset class co-movements, we plan to compare its performance against simple trading strategies such as buy-and-hold and moving average crossovers. These traditional strategies will serve as benchmarks to highlight the advanced capabilities of our model. This comparative analysis is aimed at validating the predictive capabilities of our model beyond conventional methods.

\subsection{Performance Metrics and Statistical Significance}
The evaluation will focus on key performance metrics including return on investment, Sharpe ratio, and maximum drawdown, providing a comprehensive view of both profitability and risk characteristics. Additionally, statistical tests will be conducted to ascertain the robustness of our findings, ensuring that any performance improvements are statistically significant and not due to random variations.

\subsection{Practical Implications and Contributions}
The outcomes of this evaluation will not only validate the effectiveness of our model but will also shed light on potential improvements to trading strategies in the financial markets. By demonstrating the practical benefits of our model, this research aims to contribute significantly to the evolution of financial analytics, providing actionable insights that can enhance market strategies and decision-making processes.
