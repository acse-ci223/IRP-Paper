%%%%%%%%%%%%%%%%%%%%%%%%%%%%%%%%%%%%%%%
\thesisspacing % CHAPTER
% COPY THEM IN ANY NEW CHAPTER
%%%%%%%%%%%%%%%%%%%%%%%%%%%%%%%%%%%%%%%

% Write your text here

Asset price prediction, vital for investors and policymakers, has traditionally relied on methods like Auto Regressive Integrated Moving Average (ARIMA), but recent advances in machine learning, particularly convolutional neural networks (CNNs), have shown promise in improving prediction accuracy. Kusuma et al. (2019) demonstrated that CNNs could effectively analyze historical stock data transformed into candlestick chart images, achieving high prediction accuracies for the Taiwanese and Indonesian stock markets \cite{kusuma2019using}. Sezer and Ozbayoglu (2019) converted time-series stock data into 2-D bar chart images, using CNNs to identify trading signals, outperforming traditional strategies, especially in bear markets \cite{sezer2019financial}. Zeng et al. (2021) introduced a video prediction model for economic time series, leveraging CNNs' ability to detect spatial patterns in image sequences, surpassing traditional methods like ARIMA and Prophet \cite{zeng2021deep}. Jiang (2023) used OHLC charts with CNNs to forecast stock returns, highlighting the models' ability to capture intricate patterns and their scalability across different geographies and time scales \cite{jiang2023reimagining}. These studies illustrate that CNNs, through innovative transformations of time-series data into image formats, significantly enhance asset price prediction accuracy, outperforming traditional methods and opening new avenues for integrating visual data representations in financial forecasting.