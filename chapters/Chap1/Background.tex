%%%%%%%%%%%%%%%%%%%%%%%%%%%%%%%%%%%%%%%
\thesisspacing % CHAPTER
% COPY THEM IN ANY NEW CHAPTER
%%%%%%%%%%%%%%%%%%%%%%%%%%%%%%%%%%%%%%%

Traditional methods for predicting financial markets, such as the Auto Regressive Integrated Moving Average (ARIMA) model, have been widely applied but often struggle to capture the nonlinear and intricate dynamics of financial data. The advent of machine learning, and specifically CNNs, has introduced more sophisticated techniques capable of addressing these limitations. Recent research has demonstrated the efficacy of CNNs in financial market prediction by transforming time-series data into visual formats that can better capture underlying patterns.

For instance, Kusuma et al. (2019) utilized CNNs to analyze historical stock data converted into candlestick chart images, achieving high prediction accuracy for stock markets in Taiwan and Indonesia \cite{kusuma2019using}. Sezer and Ozbayoglu (2019) further advanced this approach by transforming time-series stock data into 2-D bar chart images and applying CNNs to identify trading signals, demonstrating superior performance to traditional methods, especially during bearish market conditions \cite{sezer2019financial}. Zeng et al. (2021) expanded on these concepts by introducing a video prediction model for economic time series that leveraged CNNs' ability to detect spatial patterns in image sequences, outperforming traditional techniques like ARIMA and Prophet \cite{zeng2021deep}. Jiang (2023) also employed CNNs with OHLC (Open, High, Low, Close) charts to forecast stock returns, highlighting the model's capacity to recognize intricate patterns and adapt across different geographical and temporal scales \cite{jiang2023reimagining}. Collectively, these studies underscore the potential of CNNs to significantly improve asset price prediction accuracy, offering a marked advantage over traditional forecasting methods.
